\documentclass[a4paper,12pt]{article}
\usepackage{amsmath}
\usepackage{amssymb}
\usepackage{mathtools}
\usepackage[all]{nowidow}
\usepackage{siunitx}
\usepackage{enumerate}

\let\dotlessi\i %Copy the definition so we can safely redefine \i
\newcommand\naive{na\"{\dotlessi}ve}
\newcommand\Naive{Na\"{\dotlessi}ve}

\newcommand\norm[1]{\left\lVert#1\right\rVert}
\newcommand\abs[1]{\left\lvert#1\right\rvert}
\renewcommand{\d}{\mathrm{d}}
\newcommand{\D}[1]{\mathop{\d #1}}
\renewcommand{\i}{\mathrm{i}}
\newcommand{\e}{\mathrm{e}}
\newcommand{\Imat}{\mathrm{I}}
\newcommand{\bigO}{\mathcal{O}}
\newcommand{\transpose}[1]{{#1}^{\mathrm{T}}}

%Have to delete existing commands first.
\let\Re\relax
\let\Im\relax
\DeclareMathOperator{\Re}{Re}
\DeclareMathOperator{\Im}{Im}
\DeclareMathOperator{\trace}{tr}
\DeclareMathOperator{\J}{J}

\begin{document}

This an attempt to show an equivalence between one of Critchlow's six metrics and one of betadiver's eighteen.

Presence-absence data means there are only two ranks available for the partially ranked data, 1 and 2.
Let rank 1 be presence and rank 2 be absence.
This means that, equating betadiver's notation with Critchlow's, we have:
\begin{align*}
	a &= n_{11}\\
	b &= n_{12}\\
	c &= n_{21}\\
	d &= n_{22}\\
\end{align*}
We shall convert everything to betadiver's notation for convenience.

Kendall's tau:
\begin{align*}
	T^*\left(S\pi, S'\rho\right)
	&= \max \left\{ \Sigma_{i < i',j \geq j'} n_{ij}n_{i'j'} , \Sigma_{i \leq i',j > j'} n_{ij}n_{i'j'} \right\}\\
	&= \max \left\{ n_{11}n_{21} + n_{12}n_{21} + n_{12}n_{22} , n_{12}n_{11} + n_{12}n_{21} + n_{22}n_{21} \right\}\\
	&= \max \left\{ ac + bc + bd , ab + bc + cd \right\}\\
	&= \max \left\{ ac + bd , ab + cd \right\} + bc\\
\end{align*}

Spearman's rho:
\begin{align*}
	R^*\left(S\pi, S'\rho\right)
	&= \\
\end{align*}

\end{document}
