%New Zealand Journal of Marine and Freshwater Research

\usepackage{xpatch}
\usepackage{hyperref}%Makes citations link to the bibliography.
\usepackage[backend=biber,style=authoryear-comp]{biblatex}

\ExecuteBibliographyOptions{sorting=ynt} %Year, name, title. Ensures that in-text citations are ordered chronologically.
\assignrefcontextentries[]{*} %Assigns all citations to the global refcontext, the sorting above.
%We will use a different refcontext around the bibliography to sort that by name.

\ExecuteBibliographyOptions{dashed=false} %Ensures that two papers by the same author are listed in full in the bibliography.
\ExecuteBibliographyOptions{maxbibnames=10,minbibnames=10} %In the bibliography, the first maxnames authors are listed, then the list is truncated to minnames
\ExecuteBibliographyOptions{maxcitenames=2,mincitenames=1} %In-text, two authors are listed in full, three or more are first author then et al.
\ExecuteBibliographyOptions{citetracker=true} %For use in line below.
%\AtEveryCitekey{\ifciteseen{}{\defcounter{maxnames}{3}\clearfield{namehash}}}%Three names before andothers on first citation.
%TODO: Check letters after year for multiple papers by same author in same year.

%\ExecuteBibliographyOptions{uniquename=false,uniquelist=false}%Disables disambiguation that would violate the min/max names above.
\ExecuteBibliographyOptions{giveninits=true} %Initialises all given names.
\ExecuteBibliographyOptions{terseinits=true} %Removes periods and spaces around initials
\ExecuteBibliographyOptions{doi=false, eprint=false, url=false, isbn=false}

\DeclareNameAlias{author}{last-first} %Ensures all authors are given as "surname, given initials", rather than "given initials surname".
\DeclareNameAlias{sortname}{last-first} %Ditto for everything else
\DeclareNameAlias{default}{last-first}

%\AtEveryBibitem{\clearfield{number}} %Removes issue numbers, but leaves volume numbers.
\AtEveryBibitem{\clearfield{month}} %Removes month of publication.
\AtEveryBibitem{\clearfield{note}} %Removes the `note' field.

%\xpatchbibmacro{name:andothers} {\bibstring{andothers}} {\bibstring[\textit]{andothers}} {}{}%Italicises et al.

%Remove `In:' before the journal title for articles, but keeps it for other entries.
\xpatchbibdriver{article}{%
	\usebibmacro{in:}%
}{%
	%Just delete it; replace it with nothing.
}{}{}

\DeclareFieldFormat[article]{pages}{#1} %Removes pp. or p. from article pages ranges.
\DeclareFieldFormat[article,inproceedings,inbook,thesis]{title}{#1} %Removes quotes around article/chapter/thesis title.

%Add a period after the journal title for articles.
\xpatchbibmacro{journal+issuetitle}{%
	\usebibmacro{journal}%
	\setunit*{\addspace}%
}{%
	\usebibmacro{journal}%
	\setunit*{\adddot\addspace}%
}{}{}

%%Swap so that publisher comes *before* location.
%\xpatchbibmacro{publisher+location+date}{%
%	\printlist{location}%
%	\iflistundef{publisher}
%		{\setunit*{\addcomma\space}}
%		{\setunit*{\addcolon\space}}%
%	\printlist{publisher}%
%}{%
%	\printlist{publisher}%
%	\iflistundef{location}
%		{}
%		{\setunit*{\addcomma\space}}%%
%	\printlist{location}%
%}{}{}

%TODO: Test and if necessary correct handling of book editions numbers.
%See https://onlinelibrary.wiley.com/doi/epdf/10.1111/ele.13171 for handling in existing paper.

%Only print the book subtitle punctuation if there's a subtitle.
\xpatchbibmacro{booktitle}{%
	\setunit{\subtitlepunct}%
	\printfield[titlecase]{booksubtitle}%
}{%
	\iffieldundef{booksubtitle}
	{}
	{%
		\setunit{\subtitlepunct}%
		\printfield[titlecase]{booksubtitle}%
	}%
}{}{}

%%Remove the extra period that gets forced in after the booktitle, allowing the bibdrivers to handle it properly.
%\xpatchbibmacro{booktitle}{%
%	\newunit
%}{%
%	%Just delete it; replace it with nothing.
%}{}{}
%%Ditto; remove the extra period.
%\xpatchbibmacro{maintitle+booktitle}{%
%	\usebibmacro{booktitle}%
%	\newunit%
%}{%
%	\usebibmacro{booktitle}%
%}{}{}

%%Remove period between book title and editors.
%\xpatchbibdriver{inbook}{%
%	\usebibmacro{maintitle+booktitle}%
%	\newunit\newblock
%	\usebibmacro{byeditor+others}%
%}{%
%	\usebibmacro{maintitle+booktitle}%
%	{\setunit*{\space}}
%	\usebibmacro{byeditor+others}%
%}{}{}
%%As above, for inproceedings.
%\xpatchbibdriver{inproceedings}{%
%	\usebibmacro{maintitle+booktitle}%
%	\newunit\newblock
%	\usebibmacro{event+venue+date}%
%	\newunit\newblock
%	\usebibmacro{byeditor+others}%
%}{%
%	\usebibmacro{maintitle+booktitle}%
%	%{\setunit*{\space}}
%	%\usebibmacro{event+venue+date}%Have to knock this out to get no period between booktitle and editors. TODO: Check this is okay.
%	%{\setunit*{\space}}\newblock
%	\usebibmacro{byeditor+others}%
%}{}{}

%%Wrap the editor info in parentheses. Use the correct editor string.
%\renewbibmacro*{byeditor+others}{%
%	\ifboolexpr{
%		test {\ifnameundef{editor}}
%		and
%		test {\ifnameundef{editora}}
%		and
%		test {\ifnameundef{editorb}}
%		and
%		test {\ifnameundef{editorc}}
%		and
%		test {\ifnameundef{translator}}
%		and
%		test {\ifnameundef{commentator}}
%		and
%		test {\ifnameundef{annotator}}
%		and
%		test {\ifnameundef{introduction}}
%		and
%		test {\ifnameundef{foreword}}
%		and
%		test {\ifnameundef{afterword}}
%	}
%	{} %If we'd be printing no names at all, don't bother with the parentheses.
%	{
%		\printtext[parens]{%
%			\ifnameundef{editor}
%				{}
%				{\usebibmacro{editorstrg}%This is also changed, from `byeditor+othersstrg'.
%				\setunit{\addspace}%
%				\printnames[byeditor]{editor}%
%				\clearname{editor}%
%				\newunit
%			}%
%			\usebibmacro{byeditorx}%
%			\usebibmacro{bytranslator+others}%
%		}%
%	}%
%}

%TODO: Check book and inbook format. Article and inproceedings should be okay.

%Ensure proper citation of Dutch names:
%https://en.wikipedia.org/wiki/Van_(Dutch)#Collation_and_capitalisation

\ExecuteBibliographyOptions{useprefix=true} %Make sure it actually picks up the prefixes.

\AtBeginDocument{%
	\renewcommand*{\mkbibnameprefix}[1]{\MakeCapital{#1}} %Capitalise prefixes in citations.
}
\AtBeginBibliography{%
	\renewcommand*{\mkbibnameprefix}[1]{#1} %But not in the bibliography
}

\renewbibmacro*{begentry}{\midsentence} %Begin bibliography entries midsentence, such that there is no capitalisation of van.

\DeclareSortingNamekeyScheme{ %Sort by family name, then by prefix. So van Gardingen sorts under G, not v.
	\keypart{
		\namepart{family}
	}
	\keypart{
		\namepart{prefix}
	}
	\keypart{
		\namepart{given}
	}
	\keypart{
		\namepart{suffix}
	}
}

%Articles currently go `volume.issue`
%We want `volume(issue)`
% No dot before number of articles
\xpatchbibmacro{volume+number+eid}{%
	\setunit*{\adddot}%
}{%
}{}{}
% Number of articles in parentheses
\DeclareFieldFormat[article]{number}{\mkbibparens{#1}}

%We want a colon to introduce the page range, not a comma.
%We also don't want any space after the coolon.
\renewcommand*{\bibpagespunct}{\addcolon}

%Change the punctuation between the volume number and volume title to a comma, for books with a maintitle.
\xpatchbibmacro{maintitle+title}{%
	\setunit{\addcolon\space}%
}{%
	\setunit{\addcomma\space}%
}{}{}

%Change how editors are printed.
%No brackets, and byeditor+othersstrg comes *after* the name list.
%Translators are before editors.
%Change byeditor+othersstrg (ed. by) to editorstrg (ed.)
\renewbibmacro*{byeditor+others}{%
	\usebibmacro{bytranslator+others}%
	\ifnameundef{editor}
		{}
		{\printnames[byeditor]{editor}%
			%\clearname{editor}%If we clear this, editorstrg doesn't know to be plural
			\setunit{\addcomma\space}%
			\usebibmacro{editorstrg}%
		\newunit}%
	\usebibmacro{byeditorx}%
}
%editorstrg uses the abbreviated versions of the bibstring. We want the full ones.
\xpatchbibmacro{editorstrg}{%
	\bibstring{editor}%
}{%
	\biblstring{editor}%biblstrign always forces the long version
}{}{}
\xpatchbibmacro{editorstrg}{%
	\bibstring{editors}%
}{%
	\biblstring{editors}%biblstrign always forces the long version
}{}{}

%Join two authors with an ampersand, not 'and'
%\renewcommand*{\finalnamedelim}{\addspace\&\addspace}

%Remove the comma betwen the surname and the initials
\renewcommand*{\revsdnamepunct}{}

%Remove all italics from the bibliography with a nasty hack:
%There are some options we only want to apply in the bibliograph, so here's a nasty hack.
\let\printinnerbibliography\printbibliography %Copy the original command so we can modify it
\renewcommand{\printbibliography}{
	\begingroup
	\let\itshape\upshape %Remove all italics
	\renewcommand*{\finalnamedelim}{\addcomma\addspace} %Remove 'and' between penultimate and ultimate author
	\begin{refcontext}[sorting=nyt]%Sorts the bibliography by name, then year, then title.
		\printinnerbibliography
	\end{refcontext}
	\endgroup
}
